\section{Conclusion}\label{sec:conclusion}

    In conclusion, the integration of OutSystems, Azure, and MongoDB for large-scale applications offers a compelling framework that combines the strengths of a low-code platform, a cloud service provider, and a NoSQL database. Throughout this research, we have explored the benefits and drawbacks of each component, providing insights into their individual contributions to the integrated infrastructure stack.
    
    The advantages of MongoDB, including its scalability, schema flexibility, and performance, make it a robust choice for handling extensive data in large-scale applications. However, considerations such as the lack of native support for join operations and historical limitations in transaction support should be weighed against these strengths.
    
    Azure's cloud services bring scalability, flexibility, and global reach to the infrastructure stack, allowing applications to adapt to dynamic workloads with enhanced security measures. Nevertheless, the complexity of cost management and the learning curve associated with mastering Azure's vast array of services require careful consideration.
    
    OutSystems, as a low-code platform, facilitates rapid development, cross-platform compatibility, and seamless integration capabilities. While it empowers efficient collaboration between development and business teams, potential limitations in customization for highly complex applications and concerns about vendor lock-in should be acknowledged.
    
    In synthesizing these components, developers and enterprises must carefully evaluate the specific requirements and objectives of their projects. The success of large-scale applications depends on a judicious balance between leveraging the strengths of each technology and mitigating their respective drawbacks.
    
    This research contributes to the understanding of the integrated infrastructure stack, offering valuable insights for developers, architects, and decision-makers involved in the planning and execution of large-scale application projects. As technology continues to advance, the integration of OutSystems, Azure, and MongoDB stands as a testament to the ongoing evolution of software development methodologies and the pursuit of innovative solutions to address the challenges of the digital landscape.