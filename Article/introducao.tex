\chapter{Introdução}
\label{chap:intro}

\section{Objetivos}
\label{sec:obj}
O objetivo deste trabalho é implementar alguns métodos de otimização a problemas de otimização conhecidos e comparar os resultados de cada um. No caso deste trabalho o problema a ser resolvido é o problema da mochila, ou \textit{knapsack problem}, para a sua resolução utilizamos três algoritmos diferentes: o \textit{Bottom-UP Approach}(devolve sempre a solução ótima), o \textit{Simulated Annealing} e o \textit{Tabu Search}, os quais iremos explicar com mais detalhe no capítulo \ref{chap:dev}.

\section{Definição do Problema da Mochila}
O problema da mochila consiste conseguir organizar vários objetos com benefício \textbf{n} e peso \textbf{w} numa mochila com capacidade \textbf{W}, sendo que o objetivo é colocar os objetos na mochila de tal forma que a soma dos benefícios seja máxima.

Existem várias meta heurísticas para se aproximar da solução exata, no âmbito dos objetivos deste trabalho foram testados e implementados três métodos, duas meta heurísticas e um método para obter a solução ótima, todos são detalhadamente descritos nos próximos capítulos.


\section{Organização do Documento}
\label{sec:organ}
% !POR EXEMPLO!
De modo a refletir o trabalho feito, este documento encontra-se estruturado da seguinte forma:
\begin{enumerate}
    \item O primeiro capítulo -- \textbf{Introdução} -- Apresenta o projeto, os seus objetivos e a definição do problema a ser explorado, delineia também a respetiva organização do documento
    \item O segundo capítulo -- \textbf{Desenvolvimento} -- Expõe e explica os vários métodos e algoritmos utilizados e os seus mecanismos.
    \item O terceiro capítulo -- \textbf{Análise dos Dados} -- Aqui faz-se uma análise e comparação dos resultados obtidos através dos vários métodos utilizados.
    \item O quarto capítulo -- \textbf{Conclusões} -- Descreve uma reflexão final sobre o trabalho, bem como as tecnologias utilizadas durante do desenvolvimento da aplicação.
\end{enumerate}

