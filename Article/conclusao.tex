\chapter{Conclusões}
% OU \chapter{Trabalhos Relacionados}
% OU \chapter{Engenharia de Software}
% OU \chapter{Tecnologias e Ferramentas Utilizadas}
\label{chap:tecno-ferra}

\section{Tecnologias Utilizadas}
\label{chap3:sec:intro}
Para realizar este trabalho foi necessário a utilização de várias bibliotecas externas à biblioteca padrão do Python: \textit{time} que fornece várias funções relacionadas ao tempo.\textit{numpy}, utilizada para trabalhar com listas e funções matemáticas. \textit{radom}, utilizada para gerar números aleatórios. \textit{math}, utilizada para funções e constantes matemáticas. \textit{matplotlib}, utilizada para gerar gráficos dos dados obtidos.

\section{Conclusão}
\label{chap3:sec:concs}
Com este trabalho explorámos várias formas de resolver o problema da mochila e comparamo-los todos para ver qual dele era o mais eficiente e eficaz em vários conjuntos de dados.